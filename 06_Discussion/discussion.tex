\chapter{Diskussion}

Hier wird reflektiert, in welchem Umfang die Zielsetzungen der Arbeit erreicht werden konnten. In der Regel gelingt dies durch den Bezug auf die Evaluation.

Typischer Umfang: 1-2 Seiten.

\section{Erreichung der Ziele}

Im Kapitel Einleitung werden sechs Ziele für das Projekt eingesetzt. Durch die Evaluation wird das Projekt ist es klar, ob diese Ziele erreicht werden. Hier werden die Ergebnisse für alle Ziele erzählt:

\begin{enumerate}[labelsep=1ex]
	\renewcommand{\labelenumi}{\textbf{Z\theenumi.}}
	\item Ein Unterricht mit VR Übung wird in Moodle erstellt. Der Lerneffekt ist deutlich. Alle neun Versuchspersonen haben den Test über Infusionsvorbereitung durchschnittlich mit der Note XXX von 100 bestanden. Im Durchschnitt erwarb pro Versuchsperson bei dem Test nach der Übung XXX Punkte mehr als dem Test vor der VR Übung.
	
	\item Die VR Übung kann auf PC, Smartphone, Samsung Gear VR und HTC Vive durchgeführt werden. Mit PC, Smartphone und HTC Vive wird Studie gemacht. Durchschnittlich XXX von 5 Punkte geben, wenn die Frage gestellt wird, ob man das Gefühl hat, wirklich eine Infusion vorzubereiten.
	
	\item Keine Installation oder Aktualisierung der WebVR Applikation wird gefordert. Die WebVR Applikation kann direkt durch das eingegebene URL in Browser aufgerufen werden.
	
	\item Ring Zeiger wird auf PC, Smartphone, Gear VR eingesetzt. Raycaster wird für Gear VR Controller eingesetzt. Die Funktion der Hände wird für HTC Vive eingesetzt.
	
	\item Die Zielobjekte können während der Übung mühelos gefunden werden. Das Whiteboard bietet die Hinweise für den nächsten Schritt, was sehr hilfreich von den Versuchspersonen bewertet wird. Die Feedbacks der Objekte entsprechen die Erwartung der Versuchspersonen.
	
	\item Die VR Übung wird durch die Links in dem Unterricht aufgerufen. Und durch das gezeigte Link auf dem Whiteboard in der VR Umgebung können die Versuchspersonen wieder zurück zum Unterricht geleitet werden. XXX Versuchspersonen finden, dass die VR Übung sehr gut mit dem Unterricht verbunden wird.
	
\end{enumerate}

Zusammenfassend kann es bestimmt werden, dass das Projekt die Ziele erreicht.

